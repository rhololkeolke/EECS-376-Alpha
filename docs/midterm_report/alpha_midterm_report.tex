\documentclass[10pt,letterpaper,draft]{report}
\usepackage[utf8]{inputenc} \usepackage{amsmath} \usepackage{amsfonts}
\usepackage{hyperref}
\usepackage{ctable}
\usepackage{graphicx}
\usepackage{amssymb} \usepackage{ifdraft} \ifdraft {
  \usepackage[paperwidth=275.9mm, paperheight=279.4mm]{geometry}
  \setlength{\evensidemargin}{95mm}
  \usepackage[colorinlistoftodos,obeyDraft]{todonotes} } {
  \usepackage[left=1in,right=1in,top=1in,bottom=1in,paperwidth=275.9mm,
  paperheight=279.4mm]{geometry} \usepackage[obeyDraft]{todonotes} }

\usepackage[section]{placeins}
% \usepackage{minted}

\usepackage{longtable}

\author{Devin Schwab\\
  Mark Schultz\\
  David Jannotta\\
  Eddie Massey III}
\title{EECS 376/476 Mobile Robotics\\
  Team Alpha Midterm Report} \date{May 3, 2012}
\begin{document}
\bibliographystyle{IEEEtran}
\listoftodos
\maketitle

\tableofcontents
\newpage

\part{Overview}

\section{Team Members and Responsibilities}
\todo[inline]{Who worked on what specifically}

The members of the alpha team and their respective roles are:

Devin Schwab - Group Leader, Velocity Profiler

Mark Shultz - Steering

David Jannotta - Path Planning

Eddie Massey - Sensors


\section{Project Goals}
\todo[inline]{What we would like our robot to do at the end of the
  semester.}
The main goal for the semester is to program a robot that is capable
of taking in sensory information from the environment and then use
that information to autonomously plan goals and execute them.  To
accomplish this overall goal multiple capabilities will need to be
added to the robot.  These capabilities include:

\begin{enumerate}
\item Take in goal destinations and calculate a path based on stored
  map information and current location
\item Autocorrect path when obstacles are detected
\item Correct trajectory using a steering algorithm
\item Detect objects using the Kinect camera and autonomously plan
  actions based on the data
\end{enumerate}

\todo[color=green,inline]{Fill out more information about each subtask}

\section{Schedule and Milestones}
\todo[inline]{Not sure if this is necessary. But will include when
  things were accomplished and when things are planned to be accomplished}

\section{Tools}
\todo[inline]{E.g. Git and roslaunch files}

\subsection{Git}

In order to keep track of the different versions of our source files
we are using git.  Our git repository can be found at
\url{https://github.com/rhololkeolke/EECS-376-Alpha}.  Using Git has
allowed us to work in parallel on the different files in our source
code and merge the various edits together.  It also has the added
benefit of keeping track of every version of our source code. There
have been many times that we have tried new changes, found they didn't
work and had to revert back to an earlier working version.

In order to keep everything organized each of us has a branch for
development.  We have also have a main branch called develop in which
we merge everyone's individual commits and do final tweaks. Every
demo is merged into master and tagged for future reference.

Below is a picture of the graph of our git repository.
\missingfigure{Add an image of the repository graph}

\subsection{ROS Launch}

Roslaunch files made splitting our packages into separate nodes easier. 
Instead of having to remember the correct executable s to run and the proper order, 
we created one launch file that incorporated each of our nodes. 
Within each node package there is a directory called launch containing its own launch file called main.launch. 
There is a launch file that finds and runs each main.launch file for each package. 
One issue that we had was the launch of the look\_ahead node. In the simulator, 
look\_ahead subscribes to base\_scan, but on Jinx look\_ahead must subscribe to the topic base\_laser1\_scan. 
In order to avoid having to edit and recompile our code for the module, we simply used the remap command to echo 
the subscriptions from both base\_scan and base\_laser1\_scan. With the remap command it did not matter 
whether we were testing code on the simulator or on Jinx.

While commands from the launch files were able to assist in the versatility of our code, we did run into issues 
with launching files on Jinx. One issue was trying to use the find command to change into a package directory. The 
launch file was not able to locate the package cwru\_semi\_stable, so that we would be able to
 launch cwru\_bringup\_no\_tele.launch. While the roscd command line tool was able to locate the package, 
the find function of ROS Launch XML was not capable of locating the directory. We had to work around this by 
writing a simple script to launch cwru\_bringup\_no\_tele.launch and start\_amcl\_2ndfloor.launch.
While ROS Launch XML is a great framework there could be better documentation and high consistency for ease of use.



\subsection{GTest}

GTest is Google's C++ Unit testing
framework.\cite{http://code.google.com/p/googletest/} ROS recommends
its use for unit testing needs.  We are currently using it to test
helper classes, but would like to expand its use to other parts of our
code, including the main nodes.

The use of unit testing help solidify development by forcing us to
think about the interfaces of our classes before we start coding.
This in theory leads to cleaner, better designed code.  In addition
when adding new features or fixing bugs it is possible to break
previously implemented features.  Running the unit test suite after
each build will alert us of any regressions.

We would like to be able to write automated test suites for our code
that actually simulate the outcome in Stage. However, so far this is
just an idea.  If implemented it would allow us to systematically test
our code in the simulator and potentially automatically collect data.
For example when picking the steering constants had we had a way to
automate the simulator we could have written a short program to
autotune the robot.  Granted, it would still have to tweaked when it
ran on the real bot due to differences between simulation and the real world.

\part{Software Architecture}

Before writing the code for the demos, we started by coming up with an
overall software architecture.  The purpose of this planning was to
write each node so that as other nodes are added to our software
package we do not have to spend time rewriting old code.

This section will describe the decisions we made and what we have
planned.

\section{Node Architecture}

The main document we are basing our software architecture off of is
the following:

\FloatBarrier
\missingfigure{Attach the updated software architecture drawing}
\FloatBarrier


This figure clearly shows the different nodes we have planned, the
communication message types being passed between them, and areas of
possible future expansion.

The specifications for the nodes in the diagram and the messages being
used will be found in the rest of this section.

\section{ROS Stack Management}

We have been utilizing ROS stacks to manage our individual ROS packages.
We will try to use a single package per node. This will encourage
appropriate message passing between nodes and make sure they are not too
tightly coupled.

Stacks utilize a \texttt{stack.xml} file which contains dependencies to
be included for every package. The ROS stack will automatically build
all packages within it's root directory.

Complete documentation can be found at the
\href{http://www.ros.org/wiki/Stacks}{ROSWiki}.


\section{Node Specifications}

To help guide us while writing our nodes we wrote a small
specification for each class seen in the software architecture diagram.

\subsection{Velocity Profiler}
\todo[color=red,inline]{This needs to be updated with a more accurate discription}
This nodes is responsible for taking in a list of path segments through the PathList and PathSegment message. After receiving the specified path, velocity profiler calculates the optimum trajectory based on the velocity and acceleration constraints in specified in the path segments. In the process of computing the optimum trajectory velocity profiler will blend multiple segments into a single seamless trajectory. After the trajectory is computed velocity profiler is responsible for sending the desired velocity at each point to steering.

\subsubsection{Requirements}
\begin{enumerate}
  \item Velocity Profiler must accept PathList messages
  \item Velocity Profiler must calculate spatial trajectories from
        the accepted PathSegment messages
        \begin{enumerate}
           \item Velocity Profiler must obey all of the path constraints
                 specified in every accepted PathSegment message
        \end {enumerate}
  \item Velocity Profiler must respond to Obstacle messages from Look
        Ahead
        \begin{enumerate}
           \item Velocity Profiler must stop and wait at an obstacle for a
                 specified amount of time
           \item Velocity Profiler must alert Path Planner if an obstacle
                 has not moved after a specified time
        \end{enumerate}
   \item Velocity Profiler must publish a desired velocity based on the computed trajectory
   \item Velocity Profiler must stop when the E-Stop is enabled
   \item Velocity Profiler must resume a plan after the E-Stop is
         disabled
\end{enumerate}
      
\subsection{Look Ahead}

This node enables reactive obstacle detection. As the name implies it is responsible for ``looking'' ahead for obstacles along the path. Currently it uses the LIDAR but using either the costmap or custom code this could be expanded for other sensors such as the Kinect.

  \subsubsection{Requirements}
  \begin{enumerate}
     \item Look Ahead will parse the data from the LIDAR
     \item Look Ahead will detect objects within a minimum of 1 m along the specified path
     \item Look Ahead will exclude obstacle detection of all objects
           outside of the planned path
  \end{enumerate}

\subsection{Steering}
  \todo[color=red,inline]{Should add a little better description}
  This node is responsible for correcting the small differences
  between desired and actual heading. It takes in a desired velocity and scales
  its own steering around that value based on the segment type.

  \subsubsection{Requirements}
  \begin{enumerate}
     \item Steering will accept a list of paths
     \item Steering will correct the desired velocities from Velocity
           Profiler
     \item Steering will obey stopping commands in the desired velocity
     \item Steering will stop correcting the path when the E-Stop is pressed.
  \end{enumerate}

\subsection{Path Planner}

This node is responsible for taking in a set of path points and converting it to a list of path segments that form a path through the desired points. It is allowed to throw out conflicting data and select path segment parameters such as the speed.

  \subsubsection{Requirements}
  \begin{enumerate}
    \item Path Planner will receive a list of path points
    \item Path Planner will send out the calculated path in segments
    \item Path Planner will respond to requests for new paths
    \item Path Planner will correct its computed path based on new information in the path point list
  \end{enumerate}

\todo[color=green,inline]{Fill in new node specifications}
\subsection{Path Finder}

     \subsubsection{Requirments}
     \begin{enumerate}
       \item stuff
     \end{enumerate}
 
\subsection{Kinect}

  \subsubsection{Requirments}
  \begin{enumerate}
    \item stuff
  \end{enumerate}

\subsection{Costmap}

  \subsubsection{Requirments}
  \begin{enumerate}
    \item stuff
  \end{enumerate}
       

\subsection{A Star Search}

   \subsubsection{Requirments}
   \begin{enumerate}
     \item stuff
   \end{enumerate}

\todo[color=red,inline]{Check the definitions of new message specifications.}

\section{Custom Message Specifications}

\subsection{SegStatus}

Responsible for holding information about the status of a path
segment.  The message format is as follows:

\noindent {\bf bool segComplete}\\
\indent True if segment has been completed\\
\indent False if segment is still being executed\\
\\
{\bf uint64 seg\_number}\\
\indent Stores the number identifying the segment the message is
pertaining to.\\
\\
{\bf float64 distance}\\
\indent Stores the distance remaining on a segment.\\


Responsible for holding information about obstacles in the path
segments.  The message format is as follows:

\noindent {\bf bool exists}\\
\indent True when 1 or more obstacles detected in path\\
\indent False when 0 obstacles are detected in path\\
\\
{\bf float64 distance}\\
\indent The distance to the closest obstacle

\subsection{BlobDistance}
The Kinect Listener publishes the distance from the center of an image
to the center of a blob.

\noindent {\bf uint32 dist}\\
\indent Distance from the center of the blob to the center of the vision\\

\subsection{CentroidPoints}
Contains the list of coordinates of the centroids of the slices along the orange strap.
\noindent {\bf bool exists}\\
\indent \todo[color=red,inline]{What is the purpose of exists in centroid points}
\noindent {\bf geometry\_msgs/Point point}
\indent The point at which the centroid is.

\subsection{Goal}
Incrementally updates goals for the A $\ast$ search node.

\noindent{\bf bool new}\\
\indent True when a new goal is published.

\noindent{\bf bool none}\\
\indent Set to true when the robot should not change its present location.

\noindent{\bf geometry\_msgs/Point goal}\\
\indent The coordinates of the desired goal

\subsection{PathList}
A list of coordinates the A $\ast$ search publishes from the robots current position to the goal.

\noindent {\bf msg\_alpha/PathSegment[] segments}\\
\indent The coordinates of each point the robot must go to in order to reach the goal.

\subsection{PathSegment}
Path Segments that are generated by the Path Planner node.

\noindent {\bf int8 seg\_type}\\
\indent The segment type can be a line, arc, or spin, 1,2 or 3 respectivley.

\noindent {\bf bool relative}\\
\indent Set to true when the path is in the robot's coordinate frame.\\
\indent Set to false when the path is in the map's coordinate frame.

\noindent{\bf float64 seg\_length}\\
\indent The length of a path segment.

\noindent{\bf geometry\_msgs/Point ref\_point}\\
\indent The reference point of a path segment.

\noindent{\bf geometry\_msgs/Quaternion init\_tan\_angle}\\
\indent The initial tangent angle of a path segment.

\noindent{\bf float64 curvature}\\
\indent The curvature of a path segment.

\noindent{\bf geometry\_msgs/Twist max\_speeds}\\
\indent Maximum speed for a path segment.

\noindent{\bf geometry\_msgs/Twist min\_speeds}\\
\indent Minimum speed for a path segment.

\noindent{\bf float64 accel\_limit}\\
\indent Acceleration limit for a path segment 

\noindent{\bf float64 decel\_limit}\\
\indent Deceleration limit for this segment.

\subsection{PointList}
\noindent{\bf bool new}\\
\indent{\bf geometry\_msgs/Point[] points}






\section{Topics}

\subsection{des\_vel}
Velocity Profiler publishes the desired velocity based on the robot's
current point in space the current and next path segments.  This
velocity is then corrected by steering.

\subsection{cmd\_vel}
Steering publishes the final velocity to the robot's motors.  The
final velocity is a corrected version of the velocity in the des\_vel topic.

\subsection{base\_laser1\_scan and base\_scan}
base\_laser1\_scan is used on the robot\\
base\_scan is used in the simulator\\

\noindent Used to send out LIDAR data from the cRIO.

\subsection{seg\_status}
Velocity profiler publishes the status of its current segment on this
topic.  Nodes such as steering subscribe to it.

\todo[inline]{Check over descriptions}

\subsection{path\_seg}
Path Publisher publishes the planned path segments to this topic.  The
published segments are used by steering, velocity\_profiler and other nodes.

\subsection{point\_list}
AStar publishes a list of points for the pathplanner to generate path segments from.
\subsection{goal\_point}
The goal publisher sends a goal to the AStar node that determines the end point of the A star search.
\subsection{path}
The path list determiend by the path planner.
\subsection{obstacles}
Look ahead publishes whether or not there is an obstacle within the bounded box area directly in front of the robot.
\subsection{motors\_enabled}
Estop publishes whether or not the estop is off.


\part{Nodes}
\todo[inline]{Each person will write this section based on the one you
  have worked on}
In the software architecture section the specifications for each node
was discussed.  This section deals with the actual implementation of
each node.  This includes problems encountered, solutions found, and
future plans

\section{Velocity Profiler}

\subsection{Theory of Operation}

\subsubsection{Velocity Profiling Algorithm}

\subsection{Observations}

\subsection{Coding Procedure}

\section{Look Ahead}
\subsection{Theory of Operation}

The look ahead node is responsible for obstacle detection. The node
subscribes to lidar data and uses trigonometric functions to check for
obstacles within a bounded box.  The box's dimensions are currently 1m
in front of the robot and .25m on either side. 

The node constantly publishes to the obstacles topic using a custom
message type of ``Obstacle''.  The Obstacle message and the
subscribers of the obstacle topic are described in Custom Message
Specifications section.

In order to perform its job the Look Ahead node subscribes to the
lidar data published with message type sensor\_msgs/laserscan. This message
includes information about each laser ping's distance and intensity as
well as information about the laser range finder device itself.  In
our case, the lidar is set to do 181 pings on every sweep with spacing
of 1 degree per ping.  This works out to a sweep from -90$^\circ$ to
90$^\circ$ in the robots base coordinate frame.

When an obstacle is detected in the box the obstacles boolean in the
Obstacle message type is set to true. The algorithm also calculates
the closest of the obstacle pings and puts that number in the distance field
of the Obstacle message type. When no obstacles are detected obstacles
is set to false and distance is set to 0.0.

\subsubsection{Implementation Decisions}

To account for the possibility of more pings per sweep or sweeps of
smaller degrees we have constants that are defined globally at the top
of the file.  This gives us a single place to change fundamental
parameters of the program.  We could potentially move these into a
configuration file to prevent the need to recompile when new settings
are desired.

\subsubsection{Bounding Box Algorithm}
To determine if a ping is within the box dimensions the trigonometric
functions are used to determine the bounds of the box.  For each ping
a maximum distance is computed.  If the ping distance is larger than this
value then it is ignored. If the ping distance is less it is
considered an obstacle.  The minimum of all pings within the distance
thresholds is returned as the distance.

\FloatBarrier
\begin{figure}[h]
  \centering
  \includegraphics[height=4.5in]{Look_Ahead_Bounding_Box.png}
  \caption{Calculations for laser ping distances}
  \label{fig:lookAheadLaserPingDistances}
\end{figure}
\FloatBarrier

In this figure a laser ping from each of the two region types is shown
along with the function used to calculate the maximum distance.

\FloatBarrier
\begin{figure}[h]
  \centering
  \includegraphics[height=4.5in]{look_ahead_bounding_box_transitions.png}
  \caption{Calculations for the transition angles}
\end{figure}
\FloatBarrier

This figure shows what how the transition angles are calculated.
After the ping corresponding to one of the transition angles is surpassed,
the calculations for the next set of pings switches between the two
equations shown in figure 3.

\subsection{Observations}

\subsubsection{Lidar Problems}

Sometimes the LIDAR data would drop out. To fix this problem we would restart the computer and it would usually comeback once the computer had started back up. We are not sure what is causing the problem, however, one of our theories is that multiple people trying to run their code causes the problem. If one person starts the cwru\_bringup\_no\_tele launch file then the two will conflict and the console will print LIDAR timeout messages. After this occurs its possible that internal state variables become corrupted and so the LIDAR code on the robot fails to act properly until a restart.

Another theory is that bugs in the various groups' code such as memory leaks and segfaults leave subprocesses running in the background. These zombie processes may be causing interference. Between the wo the first theory seems more likely. Especially since the bringup launch script can be run in a background process or through a screen session.

\subsection{Unimplemented Functions}

Before spring break we had plans to update look ahead so that it could detect obstacles before going into a spin segment and so that it could detect obstacles more accurately on an arc. We had planned on taking in the path segment list and detecting when the next segment would be a spin. If we detected any obstacles in the area the robot would move through while executing the spin we would stop on the segment before the spin and wait until the obstacle moved or we replanned. This would have avoided the part of the problem with the LIDAR's blind spots.

In addition for the last demo we found that because of the very small path segment distances our bounding box algorithm, would often give responses too late to do anything but a hard stop. Our current approach is to stop the robot only if the distance to the closest obstacle is within our path segment distance. This needs to be changed to within some distance along the entire path. With the short segments published by our path planner the robot won't stop until its so close its essentially already collided with it. However, if the costmap were working this may become a moot point as that should allow us to detect all obstacles including ones from the kinect and dynamically steer around them.








\missingfigure{Attach an image of the derivation of the geometry for
  the look ahead box}

\section{Steering Node}

The Steering node is responsible for taking the desired velocities from
the velocity planner, the desired path from the path planner, and the
current position to determine correction factors to the velocity command
such that the robot does not stray from its desired path.

\subsection{Theory of Operation}

The velocity profiler's job is to decide what velocity the robot
should be moving at for each point along the path.  The steering
node's job is to take the desired velocities and add correction
factors to them.

To do so steering subscribes to the des\_vel topic published by
Velocity Profiler.  In turn Velocity Profiler subscribes to the
velocity command actually sent to the robot. Velocity Profiler uses
this information to correct its internal state so that it is inline
with the robot's actual state.  The process runs continously forming a
cycle.

\subsubsection{Steering Algorithm}

\paragraph{Steering to a Path Segment}

We used the linear steering algorithm. This algorithm takes a number
of inputs such as:

\begin{itemize}
\item
  The $x$ and $y$ coordinates of the destination.
\item
  The current $x$ and $y$ coordinates.
\item
  Tuning parameters $K_d$ and $K_\theta$
\end{itemize}
First we use the path segment coordinates to calculate the desired
heading with the following: $atan(yf-ys,xf-xs)$. Next we find a
$d_\theta$ by subtracting $heading_{dest}-heading_{curr}$. We can
prevent turning the long way by checking to see that $d_\theta$ is less
than or greater than $\pi$. Finally, we take the vector components of
the desired heading $tx=cos(heading_{dest})$, $ty=-sin(heading_{dest})$
and dot these with the vectors from the start point to the current point
$xrs*nx+yrs*ny$. We take this product and add it to $d_\theta$ to get
the final corrected heading $-K_d*offset+K_{\theta}*d_\theta$.

\paragraph{Steering to a Point}

For the strap following demo we used a bread crumb approach to path generation. Essentially instead of coming up with an entire path based on the sensor data we use only the closest desired point. Because this does not use a segment like in the other algorithm we had to modify the algorithm a bit.

When steering to a point steering simply computes a line from the robot's current position to the specified point. It currently uses a constant velocity, but this could easily be changed to accelerate and decelerate and do more complicated things.

It differs from steering to a path segment in that it assumes it has no offset from the desired path only an offset in orientation. In other words it controls only spin not forward velocity.

\subsection{Observations}

Our tests for the demo showed that the linear steering algorithm works
just fine for the relatively small imperfections in path encountered
during normal operation. However, for very large offests we did not
always get the desired results.

We found that the values of $K_d=0.5$ and $K_\theta = 1.0$ worked well
for the robot.

Before spring break we had planned on implementing a non-linear steering algorithm, however, the linear algorithm functioned just fine for our purposes throughout the entire semester. There was an attempt at making a non-linear steering algorithm, however, it was not debugged and so it crashes on start up because of indexing errors. This code can be found in a previous version in our repository.


\todo[inline]{Fill in new node information}

\section{Pathplanner Node}

What do I do?

\subsection{Theory of Operation}


\subsubsection{Path Planning Algorithm}
 This algorithm takes a number
of inputs such as:


\begin{itemize}
\item
  stuff
\item
  stuff
\item
  stuff
\end{itemize}


\subsection{Implementation}


\subsection{Observations}


\subsection{Observations}


\subsection{Future Plans}

\begin{itemize}
\item
  stuff
\item
  stuff
\item
  stuff
\end{itemize}


\section{Position Publisher Node}

This node is simply around for convenience. Instead of doing the transform from odom to map in every node we simply made a node that does the transform and publishes the result.

\subsection{Implementation}

Originally when we were converting everything to python we did not know how to use the tf library.  So we wrote this node in C++ because we had working C++ transform code from our previous steering demo.

\subsection{Observations}

This node actually turned out to be extremely useful. When setting up the robot we could simply echo the output of this node and double check that the robot was properly localized. It also meant that all the nodes were on the exact same page as to where the robot was at. The tf code will transform through time and so if the nodes were using time stamps we might have had discrepancy in transform information as the robot moved.


\section{Costmap}

The costmap node is in charge of accepting the sensor data, filtering and 
identifying obstacles and then publishing the results. This node is fairly 
autonomous and simple. It only requires specific parameters and a simple C file 
to launch it.

\subsection{Theory of Operation}

The costmap utilizes the built in Costmap2DROS class which in turn utilizes a 
Costmap2DPublisher, a Costmap2D, a VoxelCostmap2D, and an ObservationBuffer.
The costmap has two modes of operation. In global mode, the costmap will read 
the wall and terrain data from the map server and store an internal 
representation of the entire map. In local mode, the costmap functions as a 
rolling window of obstacles.

\subsubsection{Costmap Algorithm}

The costmap object takes any number of inputs specified in the configuration 
file.

\begin{itemize}
	\item The PointCloud2 input from the kinect.
	\item The LaserScan input from the lidar.
	\item the map input from the map server.
\end{itemize}

The costmap object also publishes a number of outputs containing obstacle data.

\begin{itemize}
	\item /obstacles publishes occupied cells in the costmap.
	\item /inflated\_obstacles publishes the cells in the costmap that 
		correspond to the occupied cells inflated by the inscribed radius of the 
		robot.
	\item /unknown\_space publishes the unknown cells in the costmap.
	\item /voxel\_grid publishes voxel (3D) cells when requested by the
		configuration.
\end{itemize}

\subsection{Implementation}

The costmap implementation is simple enough for us. We simply need a node that
instantiates the Costmap2DROS object. Then we must use the param directive in a 
launch file to load the parameters to the namespace of the node.

\subsection{Observations}

Unfortunately the costmap\_2d ROS package is poorly documented. The 
configuration parameters are not fully described and expected behavior is often 
not described at all. Lack of documentation and any functioning examples made 
the development process extremely slow. We spent upward of 10 days trying to 
figure out a working configuration. 
We found that simply using a global costmap was infeasible because it was too 
computationally intense and we would get obstacle updates every 10--30 
seconds. Next we tried solely using the local, rolling map costmap. This also 
gave us problems because at the time, our A* algorithm needed global wall data 
to find the best route. Due to poor documentation and lack of time, we were 
unable to implement a dual global local costmap for our final demo.

\subsection{Future Plans}

\begin{itemize}
	\item
		Implement dual global local costmap.
\end{itemize}


\section{Goal Planner Node}
The purpose of the goal planner is to pass the robot waypoints. While the path planner and point publisher are responsible for ensuring that the path is accurate, the goal planner determines the waypoints of the path itself. 

\subsubsection{Goal Planning Algorithm}
There is no perscribed algorithm for goal planning. The way points were hard coded into the program, as they are points within the hallway that the robot is calculating paths to. There were three points, the first being the corner before the objective (1), one slightly up the hall after (2), so that the robot would turn down the hall by the vending machines, then the last was in front of the door to the lab, which was our final objective (3).

\subsection{Implementation}
The predetermined goal points are imported from a csv file, then put in a list. The top element of the list is then published when necessary. The goal planner takes in data from the postion publisher, so that when it reaches a point in the map, it will begin to publish the next goal point. When the second point had been reached, and the robot was facing down the hall toward the vending machines, the kinect portion of the obstacle avoidance was swtiched off, and a color finding grid, similar to that of the strap following demo was implemented. Thus the goal, the magnet, would be detected and its point driven to, instead of being avoided.

\subsection{Observations}
This ultimately worked very well for us, as the waypoints controlled where the robot went, while the obstacle avoidance code and the path planner determined exactly how it got there. This became very important when our astar class failed to work directly before the deadline, as the robot would still reach it's goal, we just needed to find another way to determine that there were obstacles in the way.

\subsection{Future Plans}
The predetermined path is a somewhat basic. In the next itteration of the goal planning node, we would like the ability to have the goal planner dynamically determine the path, or allow a user determine the path using an interface. This would allow the robot to be used in almost area, as it would not need a gobal map, only a coordinate to drive to. While this particular demo would require map data, it is possible that GPS coordinates could be used to determine the goals.

\section{Kinect Node}

The Kinect node contains 2 main programs. A blob finder used in the first Kinect demo and the strap following code that was used in the strap following demo.

\subsection{Theory of Operation}

The Kinect node contains multiple discreet binaries that the other nodes can start up and receive information from if needed. We did this because it made sense to group all of the Kinect code in a single place, yet it also made sense to make separate executables for the wide range of vision applications available.

\subsubsection{Kinect Algorithm}

The blob finder used only image data. OpenCV was used to process the received image data. The blob finder started by converting the received image into a binary image based on the RGB thresholds set in the launch file. Then we eroded the image and dilated what was left. We then ran a segmentation and blob detection algorithm on the filtered image. From the detected blobs we selected the largest and published its centroid in terms of image pixels from the center in the horizontal direction.

The strap follower was more sophisticated then the blob follower. It filtered the image in a manner similar to the blob finder, however, it used HSV values instead of RGB values. It also utilized the depth map information. After filtering on color it took the remaining pixels and transformed their image coordinates into base\_link coordinates. If the coordinate was within certain z-thresholds it was ignored. The points that were within the z-thresholds were put into a grid. The grid can be thought of as similar to a 2D histogram. Each cell in the grid is a bin and points within the cell's bounds all get put in that same bin.

After the points were classified in the bin the centroid of each bin was calculated. Those bins with no points in it were ignored. The bin with the closest centroid was remembered and that point was published as the next point in the strap to go to.

\subsection{Implementation}
The blob finder actually used the older C openCV libraries. These were much harder to use and it took some time to figure out how to get the CMakeList to properly link all of the files. For the strap follower we made sure to use the new up-to-date C++ libraries.

Instead of the bin algorithm we originally tried to use some of the new features of the Point Cloud Library (PCL) such as voxels and octrees.  However, we could not get these to compile on the robot because of an older version of the PCL that did not have the methods we needed.

We also tried using the filter functions of the PCL library but could not figure out how to get it to compile because it used Boost pointers which do not act the same as a standard C++ pointer.

\subsection{Observations}

The blob finder had a lot of noise. We believe we should have averaged the centroids over time. In addition we had problems with using RGB for filtering. The night before the demo we had it working, but when it came time to do the demo because of slightly different lighting conditions we could not consistently pick up the strap. Luckily we had our threshold values in specified as parameters in the launch file so it was easy to change on the file.



\section{Astar Node}

The Astar node is responsible for generating the most optimal path given a list of closed points from a rolling costmap and a goal from the goal publisher.

\subsection{Theory of Operation}
After a costmap is generated it publishes to the topic inflated\_obstacles. The Astar node listens to that topic and converts each obstacle's map position into the node's own grid frame. AStar also listens to the goal\_point topic to find out the desired destination in map coordinates. The goal point is also converted to the node's grid frame. The node takes the obstacle points and adds them to a closed list. From the closed list the Astar node is able to generate the most optimal path to the goal point.


\subsubsection{AStar Algorithm}
The robot's starting position is added to the open list. The neighboring location on the map with the lowest F is added to the open list and then switched to the closed list. For each adjacent location:

\begin{itemize} 
\item If the space is an obstacle or on the closed list ignore it.
\item If it is not on the open list add it to the open list. Make the current square the parent of the square and update the cost, heuristic, and sum of the cost and heuristic.
\item If a list is on the open list already check to see if the path through this location is better than the current location.
\end{itemize}

The search stops when the the goal location is in the closed list meaning the path has been found or when the goal location is not found and there are no more points in the closed list meaning there is no path to the goal.

The algorithm then generates the path path by tracing back the pointers to the parents.

 This algorithm takes a number
of inputs such as:
\begin{itemize}
\item
  A list of closed points in map coordinates
\item
  A goal location in map coordinates
\item
  The current position of the robot in map coordinates
\end{itemize}


\subsection{Implementation}
To improve the smoothness of path generation we included a launch parameter called ``numCells'' which increases the number of grid spaces generated in the node's grid frame. 


\subsection{Lessons Learned}
While our software architecture helped us to properly distribute labor and the encapsulation of data between nodes,
we neglected to make use of a contingency plan. Our production chain was broken at the costmap and our AStar node depended
on the working costmap. If we could redo the final we would have had a backup plan to generate closed list points
through our own code. For instance we could have used lidar data to generate wall points to go into the closed list of the AStar search node.



\part{Demo Code}

\section{Demo 1}
For demo 1 our robot had to navigate the hall using only dead
reckoning.  The code for demo 1 is attached in the src/demo1
directory.

Our code has 2 functions to help main.  We separated the code for turn
in place and the code for straight segments into two functions to make
it easier to schedule a path and update the code. The direction of the
turn and straight is determined by the sign of the input. A positive
will move the robot forward or counter clockwise and negative will
move the robot backwards or clockwise.

In the spatial velocity profiler there is also a State class which
keeps track of ideal state based on the integration of velocity commands

\section{Demo 2}
The source code for demo 2 is included in src/demo2.
To start we programmed a simple E-Stop publisher for the simulator.
This allowed us to test our E-Stop resume code in the simulator
first.  However, it didn't transfer perfectly as we found out there is
a short delay between disabling E-Stop and the motors responding to
commands.

To make this work we simply added a trap statement at the top of each
of our function's main loop.  Essentially it said that if the estop
was enabled sleep for a bit and then restart the loop and try again.
This froze velocity profiler's internal values until the E-Stop was disabled.


\section{Demo 3}
The source code for demo 3 is included in src/demo3. Demo 3 functions
almost identically to Demo 2 except an obstacles trap is added after
the estop trap.  So that the order of trap preference is E-Stops then
obstacles.

The first time and obstacle is detected the robots current position,
current velocity and the object distance are used to calculate a
constant deceleration rate.  This rate is then used in the second part
of the trap until the obstacle is removed or the robot is at a
complete halt.

\section{Demo 4}
The source code for demo 4 is included in src/demo4.  This is the most
major rewrite yet.

Velocity profiler now takes in path segments from Path Publisher.  It
uses the path segments to calculate the desired distances and angles
and then executes normally. It uses the velocity commands issued by
steering to integrate in the State class.

Also the work to split State into a separate class has started.  This
new class will only integrate contributions along the path and will be
separate from velocity profiler making it easier to upgrade.  State
will also be upgrade to integrate only along the path, which will
eliminate the error of integrating the steering corrections and ending
segments early.

Steering currently uses a linear steering algorithm although an
upgrade to the non-linear algorithm is planned.  It currently
saturates its corrections about velocity profiler's desired
velocities.  The way we are currently implementing it this adds a
little bit of a deadzone close to zero.  This will be smoothed out in
the future.

Path Publisher has 5 hard coded segments.  It continously sends the
latest segment until velocity profiler signals that it is done by
setting segComplete in segStatus to true.  At that point Path
Publisher switches to the next segment and the cycle continues until
all 5 paths have been completed.

\todo[color=green,inline]{Update Demo Descriptions}

\section{Obstacle Avoidance Demo}
The source code for the obstacle avoidance demo is included in src/obstacle\_avoidance

Because of our transition to python we were not able to devote as much time to developing the avoidance demo code. Putting this on top of our new algorithms for velocity profiler meant that a lot of bugs were introduced in the obstacle avoidance. Many of these were addressed when velocity profiler was rewritten for the final demo.

For this demo we are still publishing static waypoints as in the last demo. However, when an obstacle is detected by the LIDAR in the robots path, the robot will stop for 3 seconds. If the obstacle has not moved after 3 seconds the robot assumes the obstacle is not going to move. It then sends an abort signal to the path publisher. The path publisher then creates two arcs to move the robot over to the opposite side of the hallway.  The path publisher then starts publishing short straight segments until the LIDAR no longer detects an obstacle on its preferred side of the hallway. It will then publish two more arc segments which put the robot back on its original path.

\section{Kinect Demo 1}
The source code for the first Kinect demo is included in src/kinect\_demo1

For this demo we detect blobs of orange using RGB image data from the Kinect camera. We then find the centroid of the largest blob and publish its offset from the center of the camera. We then apply a proportionality constant to that number and use it to adjust the steering spin commands.

\section{Kinect  Demo 2}
The source code for the second Kinect demo is included in src/kinect\_demo2

For this demo we filter the image using HSV values and then using depth information for points in a certain z-range.  The points are then sorted into bins and the centroid of the closest bin containing points is used as the next point in the strap path.

The steering node simply steers to that point published. If no centroid is detected steering will spin in a circle until the strap is detected again. The robot will drive up an down the strap until its batteries run out or it is stopped.

A video of this demonstration is posted at \url{http://www.youtube.com/watch?v=iIT8uX7mIvI}

\section{Final}
The source code for the final demo is included in src/final

Unfortunately we were not able to finish this demo by the time it was due. This is mostly due to fighting with the costmap.

In this demo the plan was to run a goal planner that published major waypoints and switched modes from obstacle avoidance to magnet pickup.  The waypoints from the goal planner would be used by a path finding algorithm such as A* or brushfire. The points from these algorithms would then be converted to path segments by path planner. These path segments would be turned into a trajectory by velocity profiler and corrections would be applied by steering.

In order to help pick up the magnet we made a transform to the magnet\_frame which we obtained via tape measurements. The idea was when trying to line up with the metal the robot would check if the metal was at 0.0 in the magnet frame. If not it would apply corrections as necessary.


\part{Conclusion}

This semester we accomplished a lot of complex tasks with the robot such as following a strap using vision and depth information. We also designed a general purpose framework for movement that was going to be used in the final demo had it been completed in time. In addition we developed algorithms for velocity profiler that in practice worked far better than simple integration. Over the semester our team had many accomplishments such as being the first to figure out the transformation code and being the farthest along in understanding and using the costmap. Almost to the point where it was useable. Had we more time We could have finished the entire stack.

If we were to work more on this after finishing the movement stack we would have spent time developing algorithms to do autonomous planning given a set of goals. Possible approaches to this would have been to model the environment and use a Markov Decision Process (MDP) or even to implement a type of reinforcement learning algorithm, such as Q learning, so that the robot would gain experience over time.

\section{Acknowledgements}
Our team would like to express our appreciation for the class and all that we learned this semester. We would also like to thank the other teams. Everyone was extremely helpful and it was often useful to gain insight into other approaches. We believe that inter-team cooperation was a great benefit in our learning experiences.

\bibliography{bibliography}

\appendix
\chapter{Git Statistics}


\end{document}
%%% Local Variables: 
%%% mode: latex
%%% TeX-master: t
%%% End: 
