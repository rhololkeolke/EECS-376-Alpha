\part{Demo Code}

\section{Demo 1}
For demo 1 our robot had to navigate the hall using only dead
reckoning.  The code for demo 1 is attached in the src/demo1
directory.

Our code has 2 functions to help main.  We separated the code for turn
in place and the code for straight segments into two functions to make
it easier to schedule a path and update the code. The direction of the
turn and straight is determined by the sign of the input. A positive
will move the robot forward or counter clockwise and negative will
move the robot backwards or clockwise.

In the spatial velocity profiler there is also a State class which
keeps track of ideal state based on the integration of velocity commands

\section{Demo 2}
To start we programmed a simple E-Stop publisher for the simulator.
This allowed us to test our E-Stop resume code in the simulator
first.  However, it didn't transfer perfectly as we found out there is
a short delay between disabling E-Stop and the motors responding to commands.


\section{Demo 3}
\todo[inline]{Provide info about Demo 3, its code, and what our final
  solutions were}

\section{Demo 4}
\todo[inline]{Provide info about Demo 4, its code and what our final
  solutions were}