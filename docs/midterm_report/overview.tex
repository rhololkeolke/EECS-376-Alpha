\part{Overview}

\section{Team Members and Responsibilities}

The members of the alpha team and their respective roles are:

Devin Schwab - Group Leader, Velocity Profiler/Brushfire, A* Search

Mark Shultz - Steering, Kinect Sensor, Costmap, Brushfire

David Jannotta - Path Planning, Goal Planner

Eddie Massey III - Look Ahead, A* Search\\

In general each team member was given a specific node to work on. This allowed us to work in parallel as the interfaces to our nodes were defined prior to each node being created. However, there was some overlap between who worked on what node based on group member's expertise and availability.  For instance the Brushfire node was worked on by both Mark and Devin.

In addition to the actual coding work various other tasks were assigned. This included things like the creation of launch files, the collection of bag files and documentation.

\section{Project Goals}
The main goal for this semester was the final project. The final project involved navigating through an environment with unknown obstacles to a piece of metal, picking up the piece of metal and navigating back to the start. To accomplish this the robot needed the following capabilities:

\begin{enumerate}
     \item Take in goal destination and calculate a path to the location based on stored map information and current location
     \item Detect objects using sensors such as the Microsoft Kinect
     \item Update the overall plan as new obstacles are detected
     \item Calculate a trajectory along the desired path
\end{enumerate}

\todo[color=green,inline]{Fill out more information about each subtask}

\newpage
\section{Schedule and Milestones}

\FloatBarrier
\begin{figure}[h]
\begin{longtable}{|l|l|}
  \hline {\bf Week} & {\bf Goal} \\ \hline
Weeks 1-3 & Familiarization with ROS \\ \hline
Week 4 & Dead Reckoning Demo \\ \hline
Week 5 & Velocity Profiling Demo \\ \hline
Week 6 & Estop Demo \\ \hline
Week 7 & Obstacle Detection Demo \\ \hline
Week 8 & Steering Demo \\ \hline
Week 9 & Spring Break - Code Refactoring \\ \hline
Week 10 & Obstacle Navigation \\ \hline
Week 11 & Kinect Implementation \\ \hline
Week 12 & Farmiliarization with OpenCV and Pointcloud Library \\ \hline
Week 13 & Strap Follower Demo \\ \hline
Week 14 & Costmap Integration \\ \hline
Week 15 & Path Planning Algorithims \\ \hline
Week 16 & Integration of New Nodes \\ \hline
\end{longtable}
\end{figure}
\FloatBarrier

After spring break we refactored our code in python and sought to farmiliarize ourselves with image processing libraries to be prepared for the final demo.

\section{Tools}

\subsection{Git}

In order to keep track of the different versions of our source files
we are using git.  Our git repository can be found at
\url{https://github.com/rhololkeolke/EECS-376-Alpha}.  Using Git has
allowed us to work in parallel on the different files in our source
code and merge the various edits together.  It also has the added
benefit of keeping track of every version of our source code. There
have been many times that we have tried new changes, found they didn't
work and had to revert back to an earlier working version.

In order to keep everything organized each of us has a branch for
development.  We have also have a main branch called develop in which
we merge everyone's individual commits and do final tweaks. Every
demo is merged into master and tagged for future reference.

\subsection{ROS Launch}

Roslaunch files made splitting our packages into separate nodes easier. 
Instead of having to remember the correct executable s to run and the proper order, 
we created one launch file that incorporated each of our nodes. 
Within each node package there is a directory called launch containing its own launch file called main.launch. 
There is a launch file that finds and runs each main.launch file for each package. 
One issue that we had was the launch of the look\_ahead node. In the simulator, 
look\_ahead subscribes to base\_scan, but on Jinx look\_ahead must subscribe to the topic base\_laser1\_scan. 
In order to avoid having to edit and recompile our code for the module, we simply used the remap command to echo 
the subscriptions from both base\_scan and base\_laser1\_scan. With the remap command it did not matter 
whether we were testing code on the simulator or on Jinx.

While commands from the launch files were able to assist in the versatility of our code, we did run into issues 
with launching files on Jinx. One issue was trying to use the find command to change into a package directory. The 
launch file was not able to locate the package cwru\_semi\_stable, so that we would be able to
 launch cwru\_bringup\_no\_tele.launch. While the roscd command line tool was able to locate the package, 
the find function of ROS Launch XML was not capable of locating the directory. We had to work around this by 
writing a simple script to launch cwru\_bringup\_no\_tele.launch and start\_amcl\_2ndfloor.launch.
While ROS Launch XML is a great framework there could be better documentation and high consistency for ease of use.



\subsection{Unit Testing}

Because of the complexity of each node and the system as a whole it can be extremely difficult to debug. The best practice is to prevent bugs from manifesting in the first place. Unit testing is an approach to help catch bugs before they become lost in the system. It involves writing a suite of automated tests for each function in the nodes. In general it is best to write the tests before the actual code is written as this forces the developer to think through all of the implementation decisions before writing any code. However, due to the highly dynamic nature of our code base and the compressed time in which we had to do demos we do not have a comprehensive unit testing suite.

Also, while unit testing is an excellent tool to catch bugs before they happen and prevent regressions in functionality it should be noted that they are not fool proof.  For instance the unit tests will not cover every case, and so it is possible that edge case bugs can still slip through into the final project. In addition if the unit tests have bugs then the functionality of the code it is testing cannot be properly evaluated. However, most tests are quite simple compared to the actual code implementation and so bugs that do appear in the test are usually far easier to catch than those that are in the real software.

In our project we used two unit testing frameworks: GTest and PyTest

\subsubsection{GTest}

GTest is Google's C++ Unit testing
framework.\cite{google_googletest} ROS recommends
its use for unit testing needs. At the beginning of the semester we were using it to test 
helper classes, but after the switch to python we did not use it as much. However, it would still have been useful for the nodes that dealt with the Kinect as we could not use python for those modules.

\subsubsection{PyTest}

PyTest is included with standard python distributions so it is easy to set up and run a test on any computer with python.  Python is easier to do unit testing with because of duck typing.  Duck typing refers to the phrase, ``If it looks like a duck and quacks like a duck, it must be a duck.'' \cite{python_duck_typing} This refers to python's weak typing. In strongly typed languages such as C++ in order for a method to accept a parameter it must be the type defined in the function. In python the type doesn't matter, the only thing that matters is if the parameter implements the methods that are called on the parameter. This means multiple classes can be passed into a single method as long as they all implement the same methods. This is similar to Java's interface concept, except the interfaces do not need to be explicitly specified.

The reason duck typing is useful is that we can easily write small prototyping classes for our own classes and for ROS classes. These classes can provide a very limited simulation of what the real class would provide. This means that even functions which depend on things which are difficult to control , such as network communications, can be tested by writing a mockup class.

\subsection{Integration Testing}

While Unit testing works great for functions and small pieces of classes and nodes, it does not necessarily lend itself to testing the integration of multiple nodes. Especially when the other nodes have not been implemented yet. Also the unit testing is not easily used with the simulator. Additionally, certain tests can only be run in the simulator.

To get around this problem we created small prototype nodes that would take in user input either from the command line or csv files and interact with the other nodes based on this. For instance to test the velocity profiler class a path publisher prototype was created. This prototype took in a csv file that specified a list of path segments and it published them using the PathList message. This allowed velocity profiler to be tested before the nodes on the other side of the interface were actually completed.

Our integration tests are contained in folders called integrationTests within each node's package.